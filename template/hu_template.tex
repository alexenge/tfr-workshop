% -----------------------------------------------------------------------
% === HU BERLIN BEAMER TEMPLATE ===
% -----------------------------------------------------------------------
% Put this in the YML header of your R Markdown script:
%
% output:
%   beamer_presentation:
%     includes:
%       in_header: "template/hu_template.tex"
%
% Make sure that this template is placed in the `template` sub-directory 
% of your project, together with the hu_logo.pdf and and apa.csl files.
% -----------------------------------------------------------------------

% Main color
\definecolor{hu_blue}{rgb}{0, 0.22, 0.42}
\usecolortheme[named=hu_blue]{structure}

% Title page
\addtobeamertemplate{title page}{}{\vspace{-1cm}}

% Content page header with logo
\setbeamertemplate{frametitle}{\insertframetitle\vspace{0.6cm}}
\setbeamertemplate{headline}{
    \hfill
    \includegraphics[width=2.5cm]{template/hu_logo}
    \vspace{-1.5cm}}

% Content page footer
\setbeamertemplate{footline}{
    \hfill\#\insertframenumber\hspace{0.5cm}\vspace{0.5cm}}
\setbeamerfont{footline}{size=\fontsize{10}{11}\selectfont}

% Content page margins
\setbeamersize{text margin left=1cm,text margin right=1cm}

% Use circles for bullet points
\setbeamertemplate{items}[circle]

% Code chunks and output size
\makeatletter
\@ifundefined{Shaded}{}{
    \let\oldShaded\Shaded
    \let\endoldShaded\endShaded
    \renewenvironment{Shaded}{
        \topsep=5pt\partopsep=5pt\tiny\oldShaded}{\endoldShaded}}
\makeatother
\let\oldverbatim\verbatim
\let\endoldverbatim\endverbatim
\renewenvironment{verbatim}{\tiny\oldverbatim}{\endoldverbatim}
